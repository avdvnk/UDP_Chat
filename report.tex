\documentclass{article}
\usepackage[14pt]{extsizes} 
\usepackage[utf8]{inputenc}
\usepackage[russian]{babel}
\usepackage{graphicx}
\graphicspath{{pictures/}}
\DeclareGraphicsExtensions{.pdf,.png,.jpg}
\usepackage{xcolor}
\usepackage{hyperref}
\usepackage{listings}
\usepackage{float}
\usepackage{indentfirst}
\usepackage[left=2.5cm, right=1.5cm, vmargin=2.5cm]{geometry}

\begin{document}

  \begin{center}
	   ГУАП\\
	   КАФЕДРА № 51
  \end{center}
  \begin{center}
\begin{tabular}{clcll}
& & & & \\
\multicolumn{1}{l}{ПРЕПОДАВАТЕЛЬ} & \multicolumn{1}{c}{} & & & \\ \cline{1-3}
\multicolumn{1}{|c|}{доцент, к.т.н.} & \multicolumn{1}{l|}{} & \multicolumn{1}{c|}{Линский Е.М.} & & \\ \cline{1-3}
\multicolumn{1}{|c|}{\begin{tabular}[c]{@{}c@{}}должность , уч. степень,\\ звание\end{tabular}} & \multicolumn{1}{c|}{подпись, дата} & \multicolumn{1}{c|}{инициалы, фамилия} &  &  \\ \cline{1-3}
\end{tabular}
\end{center}

\vspace{5cm}

  \begin{center}
	ОТЧЕТ О ЛАБОРАТОРНОЙ РАБОТЕ № 11\\
	СОЗДАНИЕ ПРОГРАММЫ НА ЯЗЫКЕ JAVA\\
	\vspace{1cm}
	по курсу: ТЕХНОЛОГИИ ПРОГРАММИРОВАНИЯ\\
  \end{center}
	
	\vspace{4cm}
	\begin{center}
\begin{tabular}{cccll}
& \multicolumn{1}{l}{} & & &  \\
\multicolumn{1}{l}{РАБОТУ ВЫПОЛНИЛ} & & & &  \\ \cline{1-4}
\multicolumn{1}{|c|}{СТУДЕНТ ГР. №} & \multicolumn{1}{c|}{5511} & \multicolumn{1}{c|}{} & \multicolumn{1}{c|}{Вдовенко А.} & \\ \cline{1-4}
\multicolumn{1}{|c|}{} & \multicolumn{1}{c|}{} & \multicolumn{1}{c|}{подпись, дата} & \multicolumn{1}{c|}{\begin{tabular}[c]{@{}c@{}}инициалы,\\ фамилия\end{tabular}} &  \\ \cline{1-4}
\end{tabular}
\end{center}
	\vspace{1cm}
  \begin{center}
	Санкт-Петербург 2017
  \end{center}
\thispagestyle{empty}
\newpage
	\textbf{Задание:}
	\\Написать текстовый чат для двух пользователей на сокетах. Чат должен быть реализован по принципу клиент-сервер. Один пользователь находится на сервере, второй --- на клиенте. Адреса и порты задаются через командную строку: клиенту --- куда соединяться, серверу --- на каком порту слушать. При старте программы выводится текстовое приглашение, в котором можно ввести одну из следующих команд: \\
	Задать имя пользователя (@name Vasya) \\
	Послать текстовое сообщение (Hello) \\
	Выход (@quit) \\
	Принятые сообщения автоматически выводятся на экран. Программа работает по протоколу UDP.\\

	
	\textbf{Дополнительное задание:}
	\\ Заменить серверную часть программы на бота, который загадывает число, а клиент отгадывает его дихотомией.\\
	
	
	\textbf{Инструкция:}
	\\Запуск основной части программы происходит в следующем порядке:
		\begin{itemize}
			\item {Запуск сервера, выбор порта для сервера.}
			\item {Запуск клиента, ввод адреса сервера, ввод порта.}
			\item {Ведение диалога. Использование команд смены имени и выхода.}
		\end{itemize}
	Запуск дополнительной части программы происходит также, как и основной части. Бот загадывает число, если клиент пришлет команду @startGame. Остановить угадывание числа возможно с помощью команды @stop.\\
	
	
	\textbf{Тестирование:}
	\begin{enumerate}
		\item Используем команду @quit, программа клиента завершается
		\item Используем команду @name Алексей. Сервер пришлет сообщение об успешной смене имени.
		\item Используем команду @startGame, бот ответит сообщением о том, что он загадал число и настало время его угадать.
		\item Во время игры пишем сообщения боту "Less than 50" или "More than 2". Бот отвечает на данные сообщения либо "Yes", либо "No".
		\item Когда пользователь уверен, что знает число, то отправляет его боту. В случае правильного предположения бот ответит сообщением "You won!", в противном случае прийдет ответ "Try again". 
	\end{enumerate}
\end{document}
